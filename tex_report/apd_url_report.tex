\documentclass[11pt, a4paper]{article}
\usepackage[french]{babel}
\usepackage{amsmath}
\usepackage{amsfonts}   % if you want the fonts
\usepackage{amssymb}    % if you want extra symbols
\usepackage{graphicx}  % need for figures
\usepackage{geometry}
\usepackage{tocloft}
\usepackage{secdot}
\sectiondot{section}
\sectiondot{subsection}
\renewcommand{\cftsecaftersnum}{.}
\renewcommand{\cftsubsecaftersnum}{.}		
\usepackage[utf8]{inputenc}
\usepackage{ragged2e}
\usepackage[hidelinks]{hyperref}

\begin{document}	
\begin{titlepage}
\centering
\LARGE{\textsc{Analyse et classification d'urls à des fins de segmentation automatique par groupes de ciblage}}\\
\vfill
\large Matthieu Brito Antunes\\
\large Data Scientist\\
\large Tradelab - Paris\\
\large avril 2020\\
\end{titlepage}

\newpage
\tableofcontents
\listoftables
\listoffigures
\thispagestyle{empty}

\newpage
\pagenumbering{arabic}

\section{Introduction}
\label{sec:intro}

L'ensemble des adresses web (URL) des pages visitées sur Internet peut constituer une source pertinente lorsque l'on souhaite réfléchir à une méthode de ciblage publicitaire. Les informations extraites à partir du traitement des URL permettent de comprendre les mécanismes qui régissent le regroupement d'utilisateurs, ou \textit{segmentation}. La segmentation d'utilisateurs est une étape cruciale dans le processus de ciblage publicitaire en ligne et peut s'avérer très chronophage

Ce rapport présente la construction d'un modèle d'analyse et de classification d'URL à des fins de création automatique de groupes d'utilisateurs. La première partie du rapport est centrée sur l'étude du problème auquel nous avons fait face et les enjeux associés. La seconde partie du rapport offre une vision théorique de la solution proposée pour automatiser la création de segments à partir des données d'URL collectées chaque jour chez Tradelab - Jellyfish France. La dernière partie du rapport présente la mise en place de la solution et les résultats obtenus sur des données récoltées sur une période fixe.

\section{Analyse de la structure des URL et premiers essais de segmentation automatique}

\section{Approche sémantique de l'analyse des URL et segmentation automatique par la méthode des \textit{n}-grams}

\addcontentsline{toc}{section}{References}
\bibliographystyle{plain}
\bibliography{bibliographie}

\end{document}